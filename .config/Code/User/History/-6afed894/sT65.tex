\documentclass[a4paper, 14pt]{article}
\usepackage[dvipsnames]{xcolor}
\usepackage[top=70pt,bottom=70pt,left=46pt,right=46pt]{geometry}
\usepackage{pgfplots}
\usepackage{tikz}
\usepgfplotslibrary{groupplots}
\definecolor{header}{RGB}{92,184,92}
\definecolor{defenition}{RGB}{217,83,79}
\definecolor{main_title}{RGB}{66,139,202}
\definecolor{sub_header}{RGB}{91,192,222}
\usepackage[english, russian]{babel}
\usepackage[utf8]{inputenc}
\usepackage{amsmath}
\usepackage{listings}
\usepackage{graphicx}
\usepackage{amsmath}
\usepackage{ragged2e}
\title{\textcolor{main_title}{Исследование резонансного поглощения $\gamma$ квантов}}
\author{Шмаков Владимир Евгеньевич - ФФКЭ гр. Б04-105}






\begin{document}
\maketitle



\section*{\textcolor{header}{Цель работы}}




\section*{\textcolor{header}{Теоретические сведения}}


\begin{minipage}{0.5\textwidth}

    \input{lorenz.tex}

\end{minipage}
\hfill
\begin{minipage}{0.5\textwidth}
    \raggedright
    В работе рассматривается процесс резонансного поглощения $\gamma$ квантов.
    Кванты, испущенные возбуждённым ядром налетают на поглотитель, содержащий те же
    ядра в невозбуждённом состоянии.
    \newline
    Вследствие отдачи( \textcolor{defenition}{эффекта Мессбауэра}), ядро, испускающее $\gamma$ квант, приобретает
    импульс равный по абсолютной величине импульсу $\gamma$ кванта. Вследствие этого эффекта
    линии поглощения и испускания смещаются на величину $R$ - энергию отдачи:
    \begin{equation}
        R = \frac{p^{2}}{M_{\text{я}}} = \frac{E_{\gamma}^{2}}{2 M_{\text{я}} c^{2}}
    \end{equation}
\end{minipage}




\section*{\textcolor{header}{Методика}}
\subsection*{\textcolor{sub_header}{Оборудование}}


\subsection*{\textcolor{sub_header}{Экспериментальная установка}}

\begin{figure}[h]
    \begin{center}
        \includegraphics[width = 0.4\textwidth]{experimenta_setup.png}
        \caption{Схема экспериментальной установки.}
        \label{fig:experimental_setup}
    \end{center}
\end{figure}

Схема экспериментальной установки изображена на рисунке $\ref{fig:experimental_setup}$.
Для детектирования $\gamma$ - квантов используется ФЭУ. Пересчетное устройство позволяет
устанавливать верхний и нижний пороги срабатывания. Таким образом, налетающие $\gamma$ - кванты
могут быть отсортированы по энергии(проанализирован спектр источника).

Поглотитель $\gamma$ - излучения приводится в движение посредством механизма преобразования
вращательного движения в поступательное. <<Источником>> вращательного движения является
электронный двигатель РД-09.






\section*{\textcolor{header}{Обработка экспериментальных данных}}

\begin{figure}[h]
    \begin{center}
        \input{raw_data.tex}
    \end{center}
    \caption{Экспериментальные точки, приближенные контуром Воигта( смотрите формулу $\ref{approximation_fucntion_1}$)}.
    \label{fig:raw_data}
\end{figure}

Для предварительных выводов приблизим данные контуром Воигта, часто
используемом в спектроскопии.

\begin{equation}
    \begin{aligned}
        f(x) = O - A \cdot V(x + B, \sigma, \gamma) \text{ ,} \\
        \text{ где } V(x, \sigma, \gamma) = \int_{- \infty}^{\infty} G(x', \sigma) L(x - x', \gamma) dx'
    \end{aligned}
    \label{approximation_fucntion_1}
\end{equation}

Функция $V(x, \sigma, \gamma)$ - свёртка плотностей <<центрированных>> нормального распределения и распределения Коши.
Параметр $\sigma$ - дисперсия нормального распределения, $\gamma$ - коэффициент масштаба распределения Коши.

Другими словами, величины $\sigma$ и $\gamma$ показывают степень родства искомого профиля с естественным и Лоренцевским профилем.
Как видно на рисунке $\ref{fig:raw_data}$, параметр $\gamma$ всегда больше параметра $\sigma$. ДОБАВИТЬ ВЫВОД ИЗ ЭТОГО


Найдём параметры контура точнее:
\begin{enumerate}
    \item Переведём результаты измерений в диапазон (0, 1)
          \subitem Вычтем из каждого экспериментального значения $I_{i}$ максимально достигнутое значение счета $max_{i} I_{i}$.
          \subitem Отнормируем выборку $I_{i}$
    \item Рассмотрим полученные данные с точки зрения таблично заданной плотности вероятности. Пользуясь генератором случайных чисел построим выборку с заданной плотностью. Пользуюсь статистическими методами
          оценим характерные параметры полученной выборки.
    \item Методом максимального правдоподобия найдем параметры исходного распределения. Оценим качество приближения
          пользуясь критерием $\chi^{2}$.
\end{enumerate}
\begin{figure}[h]
    \input{hist.tex}
    \caption{Нахождение параметров распределений методом максимального правдоподобия}
\end{figure}


\section*{\textcolor{header}{Вывод}}






\end{document}
