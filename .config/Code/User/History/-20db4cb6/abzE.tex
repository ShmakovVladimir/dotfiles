\documentclass[a4paper, 14pt]{article}
\usepackage[dvipsnames]{xcolor}
\usepackage[top=70pt,bottom=70pt,left=48pt,right=46pt]{geometry}
\definecolor{header}{RGB}{92,184,92}
\definecolor{defenition}{RGB}{217,83,79}
\definecolor{main_title}{RGB}{66,139,202}
\definecolor{sub_header}{RGB}{91,192,222}
\usepackage[english, russian]{babel}
\usepackage[utf8]{inputenc}
\usepackage{amsmath}
\usepackage{listings}
\usepackage{graphicx}
\usepackage{amsmath}
\title{\textcolor{main_title}{Магнитные моменты легких ядер}}
\author{Шмаков Владимир Евгеньевич - ФФКЭ гр. Б04-105}






\begin{document}
\maketitle



\section*{\textcolor{header}{Цель работы}}



\section*{\textcolor{header}{Теоретические сведения}}

Рассмотрим ядро с магнитным моментом $\boldsymbol{\mu}$ во внешнем поле с индукцией $\mathbf{B}$. Взаимодействие магнитного диполя с внешним полем приводит к появлению дополнительной энергии 
\[
E = -(\boldsymbol{\mu},\mathbf{B}).
\]
Вектор $\boldsymbol{\mu}$ ориентирован по направлению полного момента количества движения $\mathbf{M}$:
\[
\boldsymbol{\mu} = \gamma \mathbf{M},
\]
где $\gamma$ -- гиромагнитное соотношение. Вводя ядерный $g$-фактор, значение которого постоянно на одном уровне,
\[
g = \dfrac{\hbar}{\mu_\text{я}}\gamma,
\]
перепишем в виде
\[
\boldsymbol{\mu} = \dfrac{\mu_\text{я}}{\hbar}g\mathbf{M}.
\]
Квадрат вектора $\mathbf{M}$ и его проекция определяются формулами
\[
\mathbf{M}^2 = \hbar^2 I(I+1),~M_z = m\hbar,
\]
гдн $I$, целое или полуцелое число, -- спин ядра, а $m$ -- целое число, по модулю не превосходящее $I$. Тогда, проектируя $\mathbf{M}$ и $\boldsymbol{\mu}$ на направление вектора $B$, получим
\[
\mu_B = \dfrac{\mu_\text{я}}{\hbar}g M_B = \mu_\text{я} g m.
\]
Таким образом, разница между расщепившимися уровнями энергии будет
\[
\Delta E = B\Delta \mu_B = B \mu_\text{я} g.
\]
Между компонентами расщепившегося уровня могут происходить электромагнитные перезоды. Переходы с нижних компонент на верхние требуют затрат энергии и происходят лишь под действием внешнего высокочастотного поля. Энергия квантов, вызывающих электромагнитные переходы, точно определена, стало быть явление носит резонансный характер. Соответствующая частота 
\begin{equation}
\omega = \dfrac{\Delta E}{\hbar} = \dfrac{\mu_\text{я}}{\hbar}Bg.
\end{equation}
Возбуждение переходов между компонентами расщепившегося ядерного уровня носит название \textcolor{defenition}{ядерного магнитного резонанса}.
\section*{\textcolor{header}{Методика}}

\subsection*{\textcolor{sub_header}{Оборудование}}
\begin{itemize}
    \item Генератор
    \item Электромагнит 
    \item Датчик Холла
    \item Осциллограф 
    \item Катушки, создающие постоянную составляющую магнитного поля
    \item Модулирующие катушки 
\end{itemize}

\subsection*{\textcolor{sub_header}{Экспериментальная установка}}

\begin{figure}[htbp]
    \centering
    \includegraphics*[width = 0.8\textwidth]{ustanovka.jpg}
    \caption{Схема экспериментальной установки.}
    \label{fig:ustanovka}
\end{figure}


Образец 2 помещён внутрь катушки, входящей в состав генератора. 
Генератор представляет собой часть индикаторной установки 1, магнитное поле в образце создаётся с помощью электромагнита 4.
Основное магнитное поле создаётся с помощью катушек 5, питаемых постоянным током.
Величина тока регулируется реостатом $R$ и измеряется амперметром $A$. 
Небольшое дополнительное поле возбуждается модулирующими катушками 6, присоединёнными к сети переменного тока через трансформатор 3. Н
апряжение на катушках регулируется потенциометром 8.\\

Основной частью установки является генератор слабых колебаний.
Он представлет собой усилитель с положительной обратной связью, благодаря которой поддерживается непрерывная генерация.
Катушка с образцом и находящийся в ящике 1 конденсатор переменной ёмкости образуют сеточный контур генератора. 
Ёмкость конденсатора можно менять, поворачивая лимб 7. 
При наступлении ЯМР поглощение энергии в образце увеличивается, добротность сеточного контура падает и амплитуда генерации уменьшается. 
Высокочастотный сигнал с генератора усиливается и детектируется.\\

Детектирование сигнала ЯМР осуществляется с помощью промышленного прибора. Модуляция магнитного поля осуществляется с помощью небольшой катушки, частота модуляции $\approx$ 50 Гц. В зазоре электромагнита устанавливается холловский измеритель магнитного поля, а измерения ЯМР проводятся на резине (измеряется ЯМР на протонах), тефлоне (в состав входит фтор) и тяжелой воде.\\
Сигнал ядерного магнитного резонанса наблюдается на экране осциллографа.




\section*{\textcolor{header}{Обработка экспериментальных данных}}

\begin{minipage}{0.5\textwidth}
    
    \centering
    \includegraphics*[width = 0.6\textwidth]{yamr.jpg}
    \label{<label>}
\end{minipage}
%\hfil
\begin{minipage}{0.5\textwidth}

    \begin{center}
    
    \begin{tabular}{|l|l|l|l|l|}
    \hline
    Образец & $f \text{ МГц}$ & $B \text{ мТл}$ & $I \text{ А}$   \\ \hline
    Вода    & 9.811           & 218             & 0.34            \\ \hline
    Резина  & 9.201           & 215             & 0.32            \\ \hline
    Тефлон  & 10.178          & 238             & 0.40            \\ \hline
    \end{tabular}
\end{center}


\end{minipage}
\\ 
\\ 

По полученным данным удаётся определить фактор Ланде $g$ и магнитный момент ядра.

\begin{equation}
    g = \frac{h f_{0}}{\mu_{\text{я}} B}
\end{equation}



\begin{equation}
    \mu = g  \mu_{\text{я}} I
\end{equation}

Воспользовавшись формулами (2) и (3) получим:

\begin{table}[hbtp]
    \begin{center}
    
    \begin{tabular}{|l|l|l|l|l|}
    \hline
    Образец & $g$  & $\mu$ [в единицах $\mu_{\text{я}}$]   \\ \hline
    Вода    & $5.9 \pm 0.2$    & 2          \\ \hline
    Резина  & 9.201           & 1.8            \\ \hline
    Тефлон  & 10.178          & 2.24            \\ \hline
    \end{tabular}
    \caption{Результаты}
    \label{table:end}
\end{center}
\end{table}


\section*{\textcolor{header}{Вывод}}

Удалось оценить магнитные моменты ядер воды, резины и тефлона. 

Экспериментально полученные значения не сходятся с табличными данными.
Ошибка эксперимента оказалась равной $\sim 14 \%$. 




\end{document}
