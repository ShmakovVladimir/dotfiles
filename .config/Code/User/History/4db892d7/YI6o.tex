documentclass[a4paper, 12pt]{article}
\usepackage[T2A]{fontenc}
\usepackage[utf8]{inputenc}
\usepackage[english,russian]{babel}
\usepackage{amsmath, amsfonts, amssymb, amsthm, mathtools, misccorr, indentfirst, multirow}
\usepackage{wrapfig}
\usepackage{graphicx}
\usepackage{subfig}
\usepackage{enumitem}
\usepackage{adjustbox}
\usepackage{pgfplots}
\usepackage{caption}

\usepackage{geometry}
\geometry{top=20mm}
\geometry{bottom=20mm}
\geometry{left=20mm}
\geometry{right=20mm}
\newcommand{\angstrom}{\textup{\AA}}


\begin{document}
	\section{Цели и задачи исследования}
	\begin{enumerate}
		\item Ознакомление с основами теории собственной фотопроводимости полупроводников;
		\item Определение ширины запрещённой зоны кремния по спектральной зависимости собственной фотопроводимости;
		\item Определение скорости поверхностной рекомбинации.
    \end{enumerate}
    \section{Теоретическая часть}
    При воздействии на полупроводник излучения с энергией кванта $h\nu$, превышающей ширину запрещённой зоны $E_g$ в зоне проводимости, и соотвественно в валентной зоне возникают неравновесные электроны и дырки. Их появление связано с переходами электронов из валентной зоны проводимости. В результате увеличивается проводимость кристалла. Это явление называется собственной фотопроводимостью.

    В непрямозонных полупроводниках типа германия и кремния минимум зоны проводимости и максимум валентной зоны расположены в различных точках зоны Бриллюэна. В этом случае оптический переход электрона из вершины валентной зоны в минимум зоны проводимости возможен лишь при участии третьей частицы – фонона. В соответствии с законом сохранения импульса квазиимпульс такого фонона $q_{\text{ф}}\approx\hbar k_{\text{Б}}$, а энергия $\hbar\omega$ должна удовлетворять закону сохранения энергии:
    \begin{equation}
        h\nu = E_g\pm \hbar\omega_q+\hbar^2(k_n-k_c)^2/2m_n+\hbar^2k_p^2/2m_p
    \end{equation}
    где $k_n$ и $k_p$ -- начальные волновые числа электрона и дырки, а $k_c$ -- конечное волновое число электрона.

    Таким образом, край основной полосы поглощения в полупроводниках типа кремния и германия определяется непрямыми оптическими переходами, сопровождающимися поглощением и испусканием фононов. При этом для разрешённых переходов, которые доминируют в полупроводниках такого типа, коэффициент поглощения:

    \begin{equation}
        K=C\left[\frac{(h\nu-E_g+\hbar\omega_q)^2}{\exp{\frac{\hbar\omega_q}{kT}}-1}+\frac{(h\nu-E_g-\hbar\omega_q)^2}{1-\exp{-\frac{\hbar\omega_q}{kT}}}\right]
    \end{equation}
    При больших энергиях квантов $h\nu>(E_g+\hbar\omega_q)$ начинают преобладать переходы с эмиссией фононов и зависимость $K^{1/2}$ от $h\nu$ должна аппроксимироваться прямой, пересекающей ось энергии в точке $h\nu_1=E_g+\hbar\omega_q$.

    При рассмотрении случая сильного поглощения излечения в образце (оптически толстый образец), то есть при $d/K<<1$, где $d$ -- толщина образца, скорость генерации электронно-дырочных пар экспоненциально уменьшается от поверхности вглубь образца:
    \begin{equation}
        g(x)\approx K(1-R)N_0\exp{-Kx}
    \end{equation}
    где $R$ -- коэффициент отражения света, а $N_0$ -- поток квантов на единицу поверхности.

    Неоднородная германия электронов и дырок в направлении освещения приводит к появлению диффузионно-дрейфовых потоков носителей заряда: быстро диффундирующие носители (электроны) опережают медленные (дырки), что приводит к возникновению электрического поля, ускоряющего медленные носители и замедляющего быстрые и к появлению дрейфовых составляющих потоков. При этом изменение проводимости $\Delta\Sigma$ существенным образом зависит от граничных условий на поверхности образца:
    \begin{equation}
        \Delta\Sigma\sim N_0\left(1+\frac{S}{D}\frac{1}{K}\right)
    \end{equation}
    где $S$ -- скорость поверхностной рекомбинации, $D$ -- коэффициент амбиполярной диффузии.

\end{document}